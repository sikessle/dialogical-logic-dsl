\documentclass[11pt,a4paper,bibtotocnumbered]{scrreprt}		                                   

\usepackage{graphicx}
\usepackage[dvipsnames]{xcolor}
\usepackage{a4}
\usepackage[onehalfspacing]{setspace}
\usepackage[ngerman]{babel}
\usepackage[utf8]{inputenc}
\usepackage{amsmath}
\usepackage{caption}
\usepackage{listings}
\usepackage[babel,german=quotes]{csquotes}
\usepackage[printonlyused]{acronym}
\usepackage[linktoc=all, hidelinks]{hyperref}

\renewcaptionname{ngerman}\figureautorefname{Abb.}

\captionsetup[table]{belowskip=8pt}


\newcommand{\thema}{Eine domänenspezifische Sprache für die Dialogische Logik}
\newcommand{\abgabedatum}{24.07.2015}
\newcommand{\zusammenfassung}{d}


\begin{document}



% ============================ FRONT MATTER
\pagenumbering{roman}
\begin{singlespace}
\begin{titlepage}

\vspace*{-3.5cm}

\begin{flushleft}
\hspace*{-1cm} \includegraphics[width=15.7cm]{htwg-logo}
\end{flushleft}

\vspace{2.5cm}

\begin{center}
	\huge{
		\textbf{Eine Domänenspezifische Sprache für die Dialogische Logik} \\[5cm]
	}
	\Large{
		\textbf{Tobias Keh, Simon Kessler}} \\[6.5cm]
	\large{
		\textbf{Konstanz, 01.07.2015} \\[2.3cm]
	}
	
	\Huge{
		\textbf{{\sf PROJEKTARBEIT}}
	}
\end{center}

\end{titlepage}
\setcounter{page}{1}
\include{abstract}
\end{singlespace}


\tableofcontents
\listoffigures
\listoftables

\chapter*{Abkürzungsverzeichnis} 

\begin{acronym}[LAENGE]
\acro{ACM}{Association for Computing Machinery}
\end{acronym}

\cleardoublepage
% ============================ ENDE FRONT MATTER



% ============================ HAUPTTEIL THESIS
\pagenumbering{arabic}


% ============================ CHAPTER
\chapter{Einleitung} % Simon

% Entwicklung von Syntax, nicht Semantik, etc..

% ============================ CHAPTER
\chapter{Motivation} % Tobi

% ============================ CHAPTER
\chapter{Theoretische Grundlagen} 

\section{Dialogische Logik} % Tobi, Simon

\section{Domänenspezifische Sprachen} % Tobi

\subsection{Interne DSL}

\subsection{Externe DSL}

\section{Grammatiken} % Tobi
Chomsky, etc.

\section{Werkzeugunterstützung} % Tobi

% ============================ CHAPTER
\chapter{Die Sprache \enquote{Logic}} % Simon

\section{Meta-Modell}

\section{Grammatiken}

\subsection{Logische Aussagen}

\subsection{Dialogischer Dialog} % Partikelregeln

\section{Beispiele}


% ============================ CHAPTER
\chapter{Anwendungsfälle} % Simon

\section{Syntax und Semantik} % Tobi

\section{Wahr, falsch und non-liquet} % Tobi
%Behandlung von non liquet -> Semantik

\section{Strukturierte Aussagen} % Simon
%Speicherung ,Export, XML, etc...

\section{Workflow Engines} % Simon

\subsection{Liquid Democracy} % Simon

\subsection{Faktencheck} % Simon

% Whatson versteht Elementaraussagen

\section{Politischer Unterrichtsdiskurs} % Tobi



% ============================ CHAPTER
\chapter{Fazit und Ausblick} % Tobi

Zusammenfassung

Selbstkritik

Ausblick

 
% ============================ ENDE HAUPTTEIL







% ============================ LITERATURVERZEICHNIS
\begin{singlespace}
\begin{thebibliography}{9}

\bibitem{lamport94}
  Leslie Lamport,
  \emph{\LaTeX: a document preparation system},
  Addison Wesley, Massachusetts,
  2nd edition,
  1994.

\end{thebibliography}
\end{singlespace}
% ============================ ENDE LITERATURVERZEICHNIS



\end{document}

